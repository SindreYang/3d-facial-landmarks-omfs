\documentclass[class=article, crop=false]{standalone}

\usepackage{geometry}

\begin{document}


\newgeometry{
   left=34mm, right=33mm,
   top=50mm, bottom=25mm
}

\pagestyle{empty}


\section*{\centering Abstract}

Three-dimensional (3D) landmarks are used in various fields within medicine. Oral and maxillofacial surgery involves reconstructive operations on the head, face and jaw as well as facial cosmetic surgery. Landmarks are being used during the planning, follow-up and diagnostics of surgical interventions. However, placing 3D landmarks manually can be tedious and inconsistent. Artificial Intelligence (AI) assisted landmark detection can help to automate this process by making use of recent developments in the field of 3D deep learning.
This work leverages DiffusionNet for surface-learning on point clouds. DiffusionNet is a representation-independent and sampling robust network structure based on heat diffusion.
This work presents a point-wise regression method that predicts regions around landmarks with increasing activation closer to the landmark point. The initial network predicts rough landmark positions based on the raw coordinates or the heat kernel signature. A refinement network is subsequently applied to more accurately locate the landmark based on its neighbourhood sampled in high resolution. The refinement network appears effective as it improves the initial network's detection accuracy of 2.78mm to 2.22mm. Raw coordinate input shows better detection accuracy on faces and craniums that are consistently oriented in space. It was found that the isometry-invariant shape descriptor heat kernel signature yields more suitable input features for faces ``in the wild", where such assumptions cannot be made.

%A direct coordinate regression shows lower localization accuracy and only performs better than point-wise regression for high error thresholds.

\noindent
\clearpage

\end{document}
