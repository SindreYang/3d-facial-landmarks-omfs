\documentclass[class=article, crop=false]{standalone}
\usepackage[utf8]{inputenc}


\usepackage{amsfonts}
\usepackage{amsmath}
\usepackage[english]{babel}
\usepackage{booktabs}
\usepackage{caption}
\usepackage{graphicx}
\usepackage{import}
\usepackage{multicol}
\usepackage{multirow}
\usepackage[subpreambles=false]{standalone}
\usepackage{subcaption}
\usepackage{tikz}

\usepackage[linesnumbered,ruled,vlined]{algorithm2e}
\newcommand\commentfont[1]{\footnotesize\ttfamily\textcolor{blue}{#1}}
\SetCommentSty{commentfont}

\usepackage[backend=biber]{biblatex}
\addbibresource{references.bib}

\usepackage{geometry}
\geometry{
   a4paper,
   left=20mm, right=20mm,
   top=20mm, bottom=25mm
}
 
\usepackage{hyperref}
\hypersetup{
  colorlinks=true,
  linkcolor=blue,
  citecolor=black,
}

\usepackage{pgfplots}
\pgfplotsset{compat=1.3}
 
 
\setlength{\tabcolsep}{2pt} % Default value: 6pt
\renewcommand{\arraystretch}{1} % Default value: 1


\begin{document}




\section{Results}
\label{sec:results}

\newcolumntype{L}{>{\raggedright\arraybackslash}X}
\begin{table*}[!ht]

\captionof{table}{\textbf{Landmarks that are considered in this work.}
Definitions from \cite{Bidra2009} and \cite{Plooij2009}
    }
\label{table:landmark_names}
\begin{tabularx}{\textwidth}{l|l|L}
\toprule
Landmark               & Error in mm &           
\\
\midrule
Pogonion               & pg        &  \pm                                                                            \\
Nasion                 & n            & \\
Pronasale              & prn          & \\
Subnasale              & sn           & \\
Alar curvature (right) & ac-r         & \\
Alar curvature (left)  & ac-l         & \\
Exocanthion (right)    & ex-r         & \\
Endocanthion (right)   & en-r         & \\
Endocanthion (left)    & en-l         & \\
Exocanthion (left)     & ex-l         & \\
Cheilion (right)       & ch-r         & \\
Cheilion (left)        & ch-l         &                                                                   
\end{tabularx}
\end{table*}


\end{document}
